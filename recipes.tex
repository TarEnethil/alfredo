\documentclass[12pt]{article}
\usepackage[german]{babel}
\usepackage[utf8]{inputenc}
\usepackage[T1]{fontenc}
\usepackage[centering]{geometry}
\usepackage[x11names]{xcolor}
\usepackage{nicefrac}
\usepackage[nonumber,noindex,contents]{cuisine}
\usepackage{xpatch}
\usepackage{graphicx}

\title{Alfredos Rezeptbuch}
\date{}

\RecipeWidths{\textwidth}{3cm}{0cm}{1.75cm}{1.5cm}{.75cm}

% disable ingredient numbering
\makeatletter
\xpatchcmd{\Displ@ySt@p}{\arabic{st@pnumber}}{}{}{}
\makeatother

\begin{document}
\maketitle

\vspace{-1.5cm}
\centering
\includegraphics[width=0.25\textwidth]{alfredo.png}

\begin{recipe}{Pizzateig alla Pfennig-Winkelsträter}{8 Pizzen}{20--25 Minuten}
	\ingredient[500]{g}{Mehl}
	\ingredient[2]{TL}{Salz}
	Mehl und Salz in eine große Schüssel geben und kurz vermischen.
	Für einen homogeneren Teig kann das Mehl in die Schüssel gesiebt werden.
	\\	

	\ingredient[5--10]{g}{Frischhefe}
	\ingredient[300]{g}{Wasser}
	Hefe zerbröseln und in lauwarmem Wasser auflösen.
	In die Schüssel geben und mit dem Mehl verrühren.
	\\

	\freeform
	
	Teig auf die Arbeitsfläche auskippen und von Hand für 15 bis 20 Minuten kneten.
	Danach muss der Teig mindestens 6 Stunden gehen.
	Entgegen den üblichen Hinweisen sollte der Teig nicht nur mit einem feuchten Tuch abgedeckt werden, da er sonst durch die lange Gehzeit austrocknet.
	Eine Schüssel mit verschließbarem Deckel hat sich bisher bewehrt.
 	Aus den resultierenden 800 Gramm Teig werden 8 Pizzen.
\end{recipe}

\clearpage

% https://www.chefkoch.de/rezepte/1047221209572594/Pizzasauce.html
\begin{recipe}{Pizzasauce ``Dimitri''}{}{30 Minuten}
	\Ingredient{Olivenöl, kalt gepresst}
	\Ingredient{1 kleine Zwiebel, fein gehackt}
	\Ingredient{250 ml Tomaten, passiert}
	Für die Pizzasauce die Zwiebel in heißem Öl glasig werden lassen. Die passierten Tomaten dazu geben und aufkochen lassen.
	\\

	\Ingredient{2 EL Tomatenmark}
	\Ingredient{1 Zehe Knoblauch, zerdrückt}
	\Ingredient{Basilikum, Oregano, Rosmarin}
	\Ingredient{Zucker}
	\Ingredient{Meersalz, grobkörnig}
	\Ingredient{Pfeffer, geschrotet}
	Die restlichen Zutaten jetzt dazu geben und für ca. 20-25 min leicht köcheln lassen.
\end{recipe}

\end{document}
